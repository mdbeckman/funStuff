\documentclass[]{article}
\usepackage{lmodern}
\usepackage{amssymb,amsmath}
\usepackage{ifxetex,ifluatex}
\usepackage{fixltx2e} % provides \textsubscript
\ifnum 0\ifxetex 1\fi\ifluatex 1\fi=0 % if pdftex
  \usepackage[T1]{fontenc}
  \usepackage[utf8]{inputenc}
\else % if luatex or xelatex
  \ifxetex
    \usepackage{mathspec}
    \usepackage{xltxtra,xunicode}
  \else
    \usepackage{fontspec}
  \fi
  \defaultfontfeatures{Mapping=tex-text,Scale=MatchLowercase}
  \newcommand{\euro}{€}
\fi
% use upquote if available, for straight quotes in verbatim environments
\IfFileExists{upquote.sty}{\usepackage{upquote}}{}
% use microtype if available
\IfFileExists{microtype.sty}{%
\usepackage{microtype}
\UseMicrotypeSet[protrusion]{basicmath} % disable protrusion for tt fonts
}{}
\usepackage[margin=1in]{geometry}
\ifxetex
  \usepackage[setpagesize=false, % page size defined by xetex
              unicode=false, % unicode breaks when used with xetex
              xetex]{hyperref}
\else
  \usepackage[unicode=true]{hyperref}
\fi
\hypersetup{breaklinks=true,
            bookmarks=true,
            pdfauthor={},
            pdftitle={Learning R},
            colorlinks=true,
            citecolor=blue,
            urlcolor=blue,
            linkcolor=magenta,
            pdfborder={0 0 0}}
\urlstyle{same}  % don't use monospace font for urls
\setlength{\parindent}{0pt}
\setlength{\parskip}{6pt plus 2pt minus 1pt}
\setlength{\emergencystretch}{3em}  % prevent overfull lines
\setcounter{secnumdepth}{0}

%%% Use protect on footnotes to avoid problems with footnotes in titles
\let\rmarkdownfootnote\footnote%
\def\footnote{\protect\rmarkdownfootnote}

%%% Change title format to be more compact
\usepackage{titling}

% Create subtitle command for use in maketitle
\newcommand{\subtitle}[1]{
  \posttitle{
    \begin{center}\large#1\end{center}
    }
}

\setlength{\droptitle}{-2em}
  \title{Learning R}
  \pretitle{\vspace{\droptitle}\centering\huge}
  \posttitle{\par}
  \author{}
  \preauthor{}\postauthor{}
  \date{}
  \predate{}\postdate{}



\begin{document}

\maketitle


\section{Install R}\label{install-r}

\begin{itemize}
\itemsep1pt\parskip0pt\parsep0pt
\item
  You can find the most recent version of R at
  \url{http://www.r-project.org/}
\item
  The website will require you to choose a ``CRAN Mirror''. the idea is
  to find the location geographically closest to you. Right now the
  closest option is probably to scroll down to USA and choose the link
  next to ``Statlib, Carnegie Mellon University, Pittsburgh, PA''
\end{itemize}

\section{Install Development
Environment}\label{install-development-environment}

The development environment is the application that you will use to
open, edit, and execute R programs. If you already have a favorite
development environment, you can see if it's compatible with R (many of
them are). If you don't we recommend one called RStudio.

\paragraph{Installing RStudio}\label{installing-rstudio}

\begin{itemize}
\itemsep1pt\parskip0pt\parsep0pt
\item
  You need to have R installed first (see above)
\item
  RStudio can be downloaded from:
  \url{http://www.rstudio.com/products/rstudio/download/}
\item
  Select the ``installer'' link that corresponds to your operating
  system (e.g. Windows, Mac OSX). Ignore the zip/tarballs \& source code sections
\end{itemize}

\section{Tutorials}\label{tutorials}

\url{https://www.datacamp.com/courses/free-introduction-to-r}

DataCamp is a website that provides several courses available to learn R
including:

\begin{itemize}
\itemsep1pt\parskip0pt\parsep0pt
\item
  Introduction to R
\item
  Hands-On Introduction to Statistics with R
\item
  Data Manipulation
\item
  and more.
\end{itemize}

There is a subscription option, but the ``Introduction to R'' course and
parts of several others are available for free if you create an account
(i.e.~provide your email address and create a password). You can use the
tutorial even if you haven't installed R yet on your local computer.

Data Camp stops you in between modules to ask if you want to pay for the
subscription, but sometimes (like the introduction to R course) you can
bypass the subscription and continue for free. There are other tutorials
out there, but this one is recommended because it seems to strike a nice
balance of simulating real programming in a user-friendly environment
for beginners.

\section{Other Resources for R Help}\label{other-resources-for-r-help}

\begin{itemize}
\itemsep1pt\parskip0pt\parsep0pt
\item
  R help documentation

  \begin{itemize}
  \itemsep1pt\parskip0pt\parsep0pt
  \item
    If you know the name of the function you're trying to use, you can
    type something like the following into the R console to pull up the
    associated documentation: \textbf{\texttt{help(read.csv)}}
  \item
    RStudio also has an embedded help window with a search bar that will
    pull up the help documentation
  \end{itemize}
\item
  Google searches that say what you are trying to do with ``R'' or ``in
  R'' included in the search terms. Result often includes R help
  documentation, tutorial webpages, user forums (open this link in
  Google Chrome for an example: \url{http://bfy.tw/2be})
\item
  \url{http://www.statmethods.net/} - ``Quick R'' website with menus
  that you can navigate to browse through topics like Basic Statistics,
  Data Management, Basic Graphs, Advanced Graphs, etc. Each section
  includes examples with working code that you can copy and paste to
  modify for yourself.
\item
  \url{https://google-styleguide.googlecode.com/svn/trunk/Rguide.xml} -
  Google's R style guide. This document gives some pointers on how to
  write code that other people will be able to understand.
\item
  \url{http://adv-r.had.co.nz/} - Website which parallels a book called
  ``Advanced R'' by Statistics Professor and RStudio Chief Scientist
  Hadley Wickham. Although it targets a non-beginner audience, there are
  useful sections for beginners including the following sections:

  \begin{itemize}
  \itemsep1pt\parskip0pt\parsep0pt
  \item
    Vocabulary - The basics
  \item
    Style (very similar to Google Style Guide above)
  \end{itemize}
\item
  \url{https://www.coursera.org/specialization/jhudatascience/1} -
  Coursera website. Structured online courses in topics such as R
  programming and other useful skills. These can be taken free of
  charge, or you can pay to earn a verified certificate upon completion.
  Courses typically require a commitment of several hours for about 4
  weeks.
\end{itemize}



\end{document}
